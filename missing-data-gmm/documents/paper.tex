\documentclass[11pt, a4paper, leqno]{article}
\usepackage{a4wide}
\usepackage[T1]{fontenc}
\usepackage[utf8]{inputenc}
\usepackage{float, afterpage, rotating, graphicx}
\usepackage{epstopdf}
\usepackage{longtable, booktabs, tabularx}
\usepackage{fancyvrb, moreverb, relsize}
\usepackage{eurosym, calc}
% \usepackage{chngcntr}
\usepackage{amsmath, amssymb, amsfonts, amsthm, bm}
\usepackage{caption}
\usepackage{mdwlist}
\usepackage{xfrac}
\usepackage{setspace}
\usepackage[dvipsnames]{xcolor}
\usepackage{subcaption}
\usepackage{minibox}
% \usepackage{pdf14} % Enable for Manuscriptcentral -- can't handle pdf 1.5
% \usepackage{endfloat} % Enable to move tables / figures to the end. Useful for some
% submissions.

\usepackage[unicode=true]{hyperref}
\hypersetup{
    colorlinks=true,
    linkcolor=black,
    anchorcolor=black,
    citecolor=NavyBlue,
    filecolor=black,
    menucolor=black,
    runcolor=black,
    urlcolor=NavyBlue
}


\widowpenalty=10000
\clubpenalty=10000

\setlength{\parskip}{1ex}
\setlength{\parindent}{0ex}
\setstretch{1.5}


\begin{document}

\title{Reproduction and Expansion of \\"A GMM Approach For Dealing With Missing Data On Regressors" by Jason Abrevaya and Stephen G. Donald}

\author{Florian Husch, Felix Schmitz, Timothy Voeste}

\date{
    {\bf Under construction}
    \\[1ex]
    \today
}

\maketitle
\footnotetext[1]{Email: \href{mailto:s3flhusc@uni-bonn.de}{s3flhusc@uni-bonn.de}, \href{mailto:s87fschm@uni-bonn.de}{s87fschm@uni-bonn.de}, \href{mailto:s6tivoes@uni-bonn.de}{s6tivoes@uni-bonn.de}}

\begin{abstract}
    Some abstract here.
\end{abstract}

\clearpage

\begin{table}
\centering
\caption{Monte Carlo Replication Results, Design 0}
\label{table:MCReplicationResultsDesign0}
\begin{tabular}{lcccc}
\toprule
Estimation Method & Parameter & Bias & n$\times$Var & MSE \\
\midrule
Complete case method & $\alpha_0$ & -0.007 & 13.250 & 0.033 \\
 & $\beta_1$ & 0.016 & 19.861 & 0.050 \\
 & $\beta_2$ & 0.014 & 22.468 & 0.056 \\
Dummy case method & $\alpha_0$ & -0.990 & 7.664 & 0.999 \\
 & $\beta_1$ & 1.001 & 13.787 & 1.036 \\
 & $\beta_2$ & -0.997 & 12.858 & 1.026 \\
Dagenais (FGLS) & $\alpha_0$ & -0.007 & 13.250 & 0.033 \\
 & $\beta_1$ & 0.010 & 18.338 & 0.046 \\
 & $\beta_2$ & 0.001 & 21.148 & 0.053 \\
GMM & $\alpha_0$ & -0.009 & 12.234 & 0.031 \\
 & $\beta_1$ & 0.015 & 16.203 & 0.041 \\
 & $\beta_2$ & 0.004 & 19.623 & 0.049 \\
\bottomrule
\end{tabular}
\end{table}


Table \ref{table:MCReplicationResultsDesign0} on page \pageref{table:MCReplicationResultsDesign0} refers
to the simulation presented in the paper, and Table \ref{table:MCReplicationResultsDesign1} on page \pageref{table:MCReplicationResultsDesign1}
refers to the first design in the appendix.

\newpage

\appendix
\section{Appendix: Reproduction of different designs}
Some appendix text here.

\begin{table}
\centering
\caption{Monte Carlo Replication Results, Design 1}
\label{table:MCReplicationResultsDesign1}
\begin{tabular}{lcccc}
\toprule
Estimation Method & Parameter & Bias & n$\times$Var & MSE \\
\midrule
Complete case method & $\alpha_0$ & -0.001 & 2.040 & 0.005 \\
 & $\beta_1$ & 0.015 & 24.700 & 0.062 \\
 & $\beta_2$ & 0.011 & 21.468 & 0.054 \\
Dummy case method & $\alpha_0$ & -0.993 & 4.349 & 0.998 \\
 & $\beta_1$ & 1.011 & 24.348 & 1.083 \\
 & $\beta_2$ & -0.996 & 23.746 & 1.051 \\
Dagenais (FGLS) & $\alpha_0$ & -0.001 & 2.040 & 0.005 \\
 & $\beta_1$ & 0.011 & 18.166 & 0.046 \\
 & $\beta_2$ & -0.000 & 18.307 & 0.046 \\
GMM & $\alpha_0$ & 0.002 & 2.067 & 0.005 \\
 & $\beta_1$ & 0.010 & 17.625 & 0.044 \\
 & $\beta_2$ & 0.000 & 17.958 & 0.045 \\
\bottomrule
\end{tabular}
\end{table}

\begin{table}
\centering
\caption{Monte Carlo Replication Results, Design 2}
\label{table:MCReplicationResultsDesign2}
\begin{tabular}{lcccc}
\toprule
Estimation Method & Parameter & Bias & n$\times$Var & MSE \\
\midrule
Complete case method & $\alpha_0$ & -0.001 & 2.040 & 0.005 \\
 & $\beta_1$ & 0.015 & 24.700 & 0.062 \\
 & $\beta_2$ & 0.011 & 21.468 & 0.054 \\
Dummy case method & $\alpha_0$ & -0.097 & 2.011 & 0.014 \\
 & $\beta_1$ & 0.112 & 11.663 & 0.042 \\
 & $\beta_2$ & -0.996 & 10.995 & 1.019 \\
Dagenais (FGLS) & $\alpha_0$ & -0.001 & 2.040 & 0.005 \\
 & $\beta_1$ & 0.013 & 12.374 & 0.031 \\
 & $\beta_2$ & -0.002 & 12.928 & 0.032 \\
GMM & $\alpha_0$ & 0.000 & 2.069 & 0.005 \\
 & $\beta_1$ & 0.012 & 12.442 & 0.031 \\
 & $\beta_2$ & -0.002 & 13.169 & 0.033 \\
\bottomrule
\end{tabular}
\end{table}

\begin{table}
\centering
\caption{Monte Carlo Replication Results, Design 3}
\label{table:MCReplicationResultsDesign3}
\begin{tabular}{lcccc}
\toprule
Estimation Method & Parameter & Bias & n$\times$Var & MSE \\
\midrule
Complete case method & $\alpha_0$ & -0.000 & 0.204 & 0.001 \\
 & $\beta_1$ & 0.005 & 2.470 & 0.006 \\
 & $\beta_2$ & 0.003 & 2.147 & 0.005 \\
Dummy case method & $\alpha_0$ & -0.995 & 2.816 & 0.998 \\
 & $\beta_1$ & 1.002 & 14.777 & 1.041 \\
 & $\beta_2$ & -0.996 & 15.530 & 1.032 \\
Dagenais (FGLS) & $\alpha_0$ & -0.000 & 0.204 & 0.001 \\
 & $\beta_1$ & 0.004 & 2.386 & 0.006 \\
 & $\beta_2$ & 0.002 & 2.230 & 0.006 \\
GMM & $\alpha_0$ & 0.000 & 0.208 & 0.001 \\
 & $\beta_1$ & 0.004 & 2.311 & 0.006 \\
 & $\beta_2$ & 0.002 & 2.127 & 0.005 \\
\bottomrule
\end{tabular}
\end{table}

\begin{table}
\centering
\caption{Monte Carlo Replication Results, Design 4}
\label{table:MCReplicationResultsDesign4}
\begin{tabular}{lcccc}
\toprule
Estimation Method & Parameter & Bias & n$\times$Var & MSE \\
\midrule
Complete case method & $\alpha_0$ & -0.002 & 2.978 & 0.007 \\
 & $\beta_1$ & 0.008 & 7.393 & 0.019 \\
 & $\beta_2$ & 0.008 & 10.927 & 0.027 \\
Dummy case method & $\alpha_0$ & -0.993 & 5.220 & 0.999 \\
 & $\beta_1$ & 1.000 & 8.980 & 1.023 \\
 & $\beta_2$ & -0.995 & 8.672 & 1.011 \\
Dagenais (FGLS) & $\alpha_0$ & -0.002 & 2.978 & 0.007 \\
 & $\beta_1$ & 0.006 & 6.383 & 0.016 \\
 & $\beta_2$ & 0.003 & 9.666 & 0.024 \\
GMM & $\alpha_0$ & -0.000 & 2.991 & 0.007 \\
 & $\beta_1$ & 0.006 & 5.987 & 0.015 \\
 & $\beta_2$ & 0.003 & 9.509 & 0.024 \\
\bottomrule
\end{tabular}
\end{table}

\begin{table}
\centering
\caption{Monte Carlo Replication Results, Design 5}
\label{table:MCReplicationResultsDesign5}
\begin{tabular}{lcccc}
\toprule
Estimation Method & Parameter & Bias & n$\times$Var & MSE \\
\midrule
Complete case method & $\alpha_0$ & -0.007 & 13.250 & 0.033 \\
 & $\beta_1$ & 0.016 & 19.861 & 0.050 \\
 & $\beta_2$ & 0.014 & 22.468 & 0.056 \\
Dummy case method & $\alpha_0$ & -0.990 & 7.664 & 0.999 \\
 & $\beta_1$ & 1.001 & 13.787 & 1.036 \\
 & $\beta_2$ & -0.997 & 12.858 & 1.026 \\
Dagenais (FGLS) & $\alpha_0$ & -0.007 & 13.250 & 0.033 \\
 & $\beta_1$ & 0.010 & 18.338 & 0.046 \\
 & $\beta_2$ & 0.001 & 21.148 & 0.053 \\
GMM & $\alpha_0$ & -0.009 & 12.234 & 0.031 \\
 & $\beta_1$ & 0.015 & 16.203 & 0.041 \\
 & $\beta_2$ & 0.004 & 19.623 & 0.049 \\
\bottomrule
\end{tabular}
\end{table}

\begin{table}
\centering
\caption{Monte Carlo Replication Results, Design 6}
\label{table:MCReplicationResultsDesign6}
\begin{tabular}{lcccc}
\toprule
Estimation Method & Parameter & Bias & n$\times$Var & MSE \\
\midrule
Complete case method & $\alpha_0$ & 0.006 & 4.590 & 0.012 \\
 & $\beta_1$ & -0.001 & 2.721 & 0.007 \\
 & $\beta_2$ & -0.004 & 6.198 & 0.016 \\
Dummy case method & $\alpha_0$ & -1.000 & 5.113 & 1.013 \\
 & $\beta_1$ & 1.008 & 7.932 & 1.036 \\
 & $\beta_2$ & -0.990 & 8.125 & 1.000 \\
Dagenais (FGLS) & $\alpha_0$ & 0.006 & 4.590 & 0.012 \\
 & $\beta_1$ & -0.008 & 3.526 & 0.009 \\
 & $\beta_2$ & -0.006 & 6.416 & 0.016 \\
GMM & $\alpha_0$ & 0.006 & 4.431 & 0.011 \\
 & $\beta_1$ & -0.003 & 2.656 & 0.007 \\
 & $\beta_2$ & -0.004 & 6.009 & 0.015 \\
\bottomrule
\end{tabular}
\end{table}

\begin{table}
\centering
\caption{Monte Carlo Replication Results, Design 7}
\label{table:MCReplicationResultsDesign7}
\begin{tabular}{lcccc}
\toprule
Estimation Method & Parameter & Bias & n$\times$Var & MSE \\
\midrule
Complete case method & $\alpha_0$ & 0.007 & 6.466 & 0.016 \\
 & $\beta_1$ & 0.001 & 4.685 & 0.012 \\
 & $\beta_2$ & -0.004 & 15.196 & 0.038 \\
Dummy case method & $\alpha_0$ & -1.000 & 4.335 & 1.010 \\
 & $\beta_1$ & 1.010 & 10.879 & 1.048 \\
 & $\beta_2$ & -0.986 & 11.101 & 1.000 \\
Dagenais (FGLS) & $\alpha_0$ & 0.007 & 6.466 & 0.016 \\
 & $\beta_1$ & -0.008 & 6.068 & 0.015 \\
 & $\beta_2$ & -0.007 & 14.519 & 0.036 \\
GMM & $\alpha_0$ & 0.005 & 6.048 & 0.015 \\
 & $\beta_1$ & -0.000 & 4.243 & 0.011 \\
 & $\beta_2$ & -0.004 & 13.845 & 0.035 \\
\bottomrule
\end{tabular}
\end{table}

\begin{table}
\centering
\caption{Monte Carlo Replication Results, Design 8}
\label{table:MCReplicationResultsDesign8}
\begin{tabular}{lcccc}
\toprule
Estimation Method & Parameter & Bias & n$\times$Var & MSE \\
\midrule
Complete case method & $\alpha_0$ & -0.082 & 2276.703 & 5.698 \\
 & $\beta_1$ & 0.079 & 2101.930 & 5.261 \\
 & $\beta_2$ & 0.002 & 163.816 & 0.410 \\
Dummy case method & $\alpha_0$ & -0.986 & 33.374 & 1.055 \\
 & $\beta_1$ & 0.987 & 122.415 & 1.279 \\
 & $\beta_2$ & -0.997 & 64.038 & 1.153 \\
Dagenais (FGLS) & $\alpha_0$ & -0.082 & 2276.703 & 5.698 \\
 & $\beta_1$ & 0.119 & 3180.107 & 7.964 \\
 & $\beta_2$ & 0.051 & 2917.172 & 7.296 \\
GMM & $\alpha_0$ & -0.032 & 63.005 & 0.159 \\
 & $\beta_1$ & 0.043 & 70.560 & 0.178 \\
 & $\beta_2$ & 0.009 & 56.225 & 0.141 \\
\bottomrule
\end{tabular}
\end{table}



\end{document}
